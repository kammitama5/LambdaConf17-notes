\documentclass{article}
\usepackage[utf8]{inputenc}

\title{Lambda Conf Notes}
\author{Krystal Maughan }
\date{May 25th 2017}

\begin{document}

\maketitle

\mathversion{bold}{$ Notes: $}
\\
\\
\mathversion{normal}
\\
Over 400 attendees were at this LambaConf.
Keynote by Ed Lattimore, who spoke on Fear.
\\
\section{Introduction} 
Category Theory by David Spivak
\\
\\
Set 
\\
How to compose Functions
\\
Monad/ Bind Return
\\
\\
Branches of mathematics need to talk to each other
\\
\\
\section{Section}
Category of Sets
\\
Category of Vect
\\
Category of Hask
\\
Category of Poset
\\
Category of Topological Spaces 
\\
Set: bag full of dots
\\
A $\rightarrow_{f}$ B
\\
\\
Identity function: C $\rightarrow$ C
\\
\\
\\
\\
\\
\section{Bijective}
People $\rightarrow$ Place 
\\
People $\leftarrow$ Place
\\
\\
Initial Set: Set with no elements 
\\
The Empty set
\\
For any A, there is a unique set from
\\
$\emptyset$ $\rightarrow$ A
\\
\\$\forall$ x : X $\exists$! a: A such that f(x) = a
\\
\section{Terminal Set}
From A to B by 
\\
$Hom_{set}(A, B)$ or Set(A, B)
\\
Hom($\emptyset$, A) = 1
\\
\\
\section{Universal Properties}
Given Sets A and B, what is special about their product
\\
A x B = [(a, b)| a $\epsilon$ A, b $\epsilon$ b ]
\\
\section{Coproducts}
For any A, B,
\\
A + B is a set such that A+B $\rightarrow$ X
\\
\\
a = [1], b = [1, 2]
\\
a + b
\\
(a, 1), (b, 1), (b, 2)
\\
Isomorphic: way of getting from one to the other
\\
\section{Exponentials}
Think of Exponentials as Function Objects
\\
(n, n, n) and (m, m)
\\
How many different ways to map from one to the other? 
\\
Number of possibilities = $2^{3}$
\\
Hom(A,B) is a set denoted
$B^{A}$
Type A $\rightarrow$ B
\\
Bijection is equivalence relation
\\
Reflective, Transformative, Equivalence
\\
Bijection = Isomorophic in set
\\
\\
Exponential satisfy universal property
\\
For any A,B, C
\\
Hom(A x B, C) $\equiv$ Hom(A, $C^{B}$)
\\
currying
\\
$C^{A \times B}$ $\equiv$ $(C^{B})^{A}$
\\
\\
\section{General}
What is a category?
\\
Def: A category C consists of: 
\\
(Ob(C))
\\
Elements of a set are called Objects
\\
For every c, d $\epsilon$ $\leftarrow$ Ob(c), a set
\\
$Hom_{C}(c,d)$ elements are called morphisms
\\
f $\epsilon$ Hom(c, d) write f: c $\rightarrow$ d c $\rightarrow^{f}$ d
\\
For every c $\epsilon$ Ob(c), a choice $id_{c} \epsilon$ $Hom_{c}(c, c)$
\\
returns self
\\
For every f $\leftarrow$ Hom(c,d), g $\epsilon$ Hom(d,e)
an element f; g $\epsilon$ Hom(c, e)
\\
\\
satisfying
\\
A. for any f: c $\rightarrow$ d 
\\
$id_{c}$ ; f = f = f (identity)
\\
Unital
\\
\\
B. Associative 
\\
Discrete Category
\\
Satisfies all the rules 
\\
category each element must map to itself
\\
Any set can be regarded as a discrete category
\\
Any poset (partially ordered set) can be regarded as a category
\\
eg. if I am level 5 clearance I also have clearance for 4, 3, 2 and 1
\\
$\forall$ A $\epsilon$ P, a $\leq$ a (reflexive)
\\
(so I have privileges less than or equal to myself)
\\
\\
A poset is a pre-order P $\leq$ such that a $\leq$ b, b $\leq$ a $\rightarrow$ a = b
\\
A poset / preorder can be regarded as category:
\\
it is one where every $Hom_{c}$(a, b) has at most one element
\\
$a \rightarrow b$
\\
$a \leq b$
\\
Initial object: map from one thing to another
\\
Everything maps to uniquely -> called Terminal object
\\
\section{Monoid}
Any monoid can be regarded as a category
\\
A monoid consists of a set M
\\
an element e $\leftarrow$ M
\\
a function * M $\times$ M $\rightarrow$ M
\\
such that e * m = m * e for all m
\\
(m * n) * p = m * n + p
\\
eg. (List A, [ ], append)
\\
\\
Path through a graph are called free categories
\\
\section{Lunch}
Lunch
\section{Functors, Natural Transformations Adjunctions}
Functor: Maps between categories
\\
Def: Let C and D be categories
\\
A functor F from C to D, write F : C $\rightarrow$ D
\\
A function F : Ob(C) $\rightarrow$ Ob(D)
\\
For every $c_{1}$, $c_{2}$ $\epsilon$ Ob(C) a function
\\
F:$Hom_{C}(c_{1}, c_{2}) \rightarrow Hom_{C}(F(c_{1}), F(c_{2})$
\\
such that 
A. for any C $\epsilon$ Ob C, F($id_{c}$) = id
\\
B. For any $c_{1} \rightarrow c_{2} \rightarrow c_{3}$
F(f;g) = F(f) ; F(g)
\\
\\
monotone : preserves less than
\\
A functor P $\rightarrow$ Q is a monotone map P $\rightarrow$ Q.
\\
P $\leq$ P $\rightarrow$ f(p) $\leq$ f(p)
\\
return : A $\rightarrow$ List A
\\
Let M, N be monoids
\\
A functor M $\rightarrow$ N
\\
id $\rightarrow$ id
\\
F($m_{1} \times m_{2}$) = F$(m_{1}) \times F(m_{2})$
\\
Cat is a category
\\
Ob(Cat) = all categories
\\
$Hom_{cat}$ (C, D) = F: C $\rightarrow$ D functors
\\
A functor F: C $\rightarrow$ D is called faithful if
$\forall$ $c_{1}, c_{2}$ if F($Hom_{C}(c_{1}, (c_{2}) \rightarrow F(c_{1}, c_{2})$ 
\\
is injective ie if f, g : $c_{1} \rightarrow c_{2}$
\\
are such that F(f) = F(g), then f = g
\\
\section{Natural Transformations}
A natural transformation
\\
Let C, D be cats
\\
F,G:C $\rightarrow$ D be functors
\\
A natural transformation p : F $\rightarrow$ G consists of 
\\
for each c $\exists$ Ob(C), morphism $P_{c}$ : F(c) $\rightarrow$ G(c) in D
\\
such that
\\
for every f : $c_{1} \rightarrow c_{2}$ in C
\\
F$(c_{1}) \rightarrow_{F(f)} F(c)$
\\
G$(c_{1}) \rightarrow_{G(f)} G(c_{2})$
\\
\section{Graph Homomorphisms}
Let G, H be graphs
\\
A graph homomorphism
\\
f: G $\rightarrow$ H is 
\\
a function G(v) $\rightarrow$ H(v)
\\
a function G(A) $\rightarrow$ H(A)
\\
such that 
\\
respecting source and target
\\
database homomorphism
\\
For every two categories C and D 
\\
there is a category Fun(C, D) whose objects are the Functors
\\
F : C $\rightarrow$ D
\\
$Hom_{(C,D)}$(F,G) = p: F $\rightarrow$ G 
\\
f: $c_{1} \rightarrow c_{2}$ in C
\\
\\
Look at FQL (Topos)
\\
Algebraic Databases 5/26 last session
\\
\\
\section{Adjunctions}
Adjoints(Adjunctions)
\\
Let C, D be cats, 
\\
C $\rightarrow^{L}$ D and D $\rightarrow^{R}$ C functors
\\ 
we say L is the left adjoint of R
\\
R is right adjoint of L
\\
if there is a natural isomorphism
if for all c $\epsilon$ Ob(C), d $\epsilon$ Ob(D), there is an iso
\\
Hom(L(c), d)  $\equiv$ Hom (c, R(d))
\\
\\
Sat 11:30am Adjunctions
\\
\\
currying is an example from adjunctions
\\
\\
Fix a set A
\\
Let Set $\rightarrow^{L_{A}}$ Set
Set $\rightarrow^{R_{A}}$ Set
\\
For any Set X $\epsilon$ Set
\\
$L_{A}(X): = A \times X$ is a functor
\\
$R_{A}(Y): = Y^{A}$
\\
$R_{A}(L_{A}(X))$ = (X $\times A)^{A}$
\\
$L_{A}(R_{A}(X)) = X^{A} \times A$
\\
$Hom_{set}(L_{A}(X), Y) \equiv Hom_{set} (X, R_{A}(Y))$
\\
functors X $\times$ A $\rightarrow$ Y $\equiv$ X $\rightarrow$ $Y^{A}$
\\
\\
Set $\rightarrow$ Monoid
\\
Free Monoid
\\
X $\rightarrow$ (List(X), [ ], append)
\\
M $\leftarrow$ (M, e, *)
\\
$Hom_{Monoid}$(Free(X), M) $\equiv$ $Hom_{set}(X, U(M))$
\\
List(X) $\rightarrow$ M
\\
f:X $\rightarrow$ U(M)
\\
Triangle Laws
\\
\\
just: X $\rightarrow$ X + 1 
\\
Just
\\
\\
X + 1 $\rightarrow$ X 
\\
From Just
\\
X $\rightarrow$ List(X) (unit)
\\
Singleton
\\
\\
List(X), [ ], append) $\rightarrow$ M  (counit)
\\
Singleton $\rightarrow$ goes from type to list
\\
Fold goes in other direction
\\
\\
\section{Galois connections}
If C, D are posets
\\
Then a Galois connection is 
\\
a pair of adjoint functors
\\
$\forall$ c $\epsilon$ C, d $\epsilon$ D
\\
L(c) $\leq$ d $\iff$ c $\leq$ R(d)
\\
\\
P(A) $\rightarrow^(im(f))$ P(B)
\\
P(B) $\leftarrow^{f^{*}}$ P(A)
\\
\\
\section{Monads}
Let C be a category. 
\\
A monad on C consists of:
\\
1. A Functor M : C $\rightarrow$ C
\\
2. A Natural transformation from $id_{C} \rightarrow M$
\\
A natural transformation $\mu$ : M $\circ$ M $\rightarrow$ M 
\\
\\
List: Set $\rightarrow$ Set
\\
$Singleton_{X} X \rightarrow List X$
\\
flatten: List [List(X)] $\rightarrow$ List(X)
\\
\\
$[ [a, b], [ ], [b, c] ]$
\\
\\
C$\rightarrow_{L}$ D
\\
D$\leftarrow_{R}$ C
\\
$\mu$ : $id_{e}$ $\rightarrow$ RL
\\
$\epsilon$ : LR $\rightarrow$ $id_{D}$
\\
Associated Monad on C:
\\
(M = RL, $\mu$, R $\epsilon$ L)
\\
R $\tau$ RL $\rightarrow$ RL
\\
\\
Associated Co-monad
\\
N : = (LR, $\epsilon$, L $\mu$ R)
\\
\\
Suppose C is a cat and (M, $\nu$, $\mu$) is a monad.
\\
\section{Kleisli category}Ob(Kl(C)) = Ob(C)
\\
$Hom_{Kl(C)} (c, d) = Hom_{C}(c, Md)$
\\
$\eta$ : c $\rightarrow$ Mc
\\
$c \rightarrow Md$
\\
$d \rightarrow Me$
\\
c $\rightarrow$ Md $\rightarrow$ M$M_{e}$ $\rightarrow_{\mu}$ Me
\\
\section{Eilenberg Moore}
Given C, M, $\eta$, $\mu$
\\
Ob: M-algebras
\\
an object $c \epsilon C$
\\
a morphism h:$M_{c} \rightarrow c$
\\
such that get out of monad using join of monad
\\
\\
M : C $\rightarrow$ C (endofunctor)
\\
an M-algebra is 
\\
a c $\epsilon$ Ob C
\\
a morphism Mc $\rightarrow$ c
\\
A morphism of M-algebras (c, h), ($c^{'}$, $h^{'}$)
\\
consists of a morphism $c \rightarrow^{f} c^{'}$
such that
\\
Mc $\rightarrow^{h} c \rightarrow c^{'}$
$M^{f} \rightarrow Mc^{'} \rightarrow^{h^{'}}c^{'}$
\\
A coalgebra is the choice of an X $\epsilon$ Ob(C) taking 
\\
h: X $\rightarrow$ MX
\\
\\
catamorphism
\\
\\
Stream(X)
\\
Stream(A) $\rightarrow$ (A $\times$ Stream A) + 1
\\
X $\rightarrow$ A $\times$ X + 1
\\


\end{document}
