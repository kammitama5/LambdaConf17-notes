\documentclass{article}
\usepackage[utf8]{inputenc}
\usepackage[T1]{fontenc}
\title{Category Theory Notes}
\author{Krystal Maughan }
\date{May 30th 2017}

\begin{document}

\maketitle

\mathversion{bold}{Catgory Theory}
\mathversion{normal}
\section{Eilenberg-MacLane Spaces}
was regarded as abstract nonsense, dogmatic
working in Topology
\section{Grothendieck}
60s, algebraic geometry
\section{Lawvere, Tierney}
logic, alternative foundations for mathematics
\\
\section{Category Theory}
Homological Algebra
\\
new kinds of algebraic objects
\\
combinatorial
\\
simplicial
\\
graph
\\
computer science
\\
John Baez
\\
Braid diagrams
\\
higher and enriched categories 
\\
\section{What is Category Theory?}
Category Theory vs Set Theory
\\
How do objects behave and relate to each other?
\\
relations between things
\\
Algebraic co-product
\\
common vocabulary among areas in mathematics
\\
categorical vocabulary
\\
Duality
\\
Comma Categories
\\
Yoneda Lemma
\\
Colimits and Limits
\\
Adjunctions
\\
Monads
\\
Beck's Theorem
\\
Abelian Categories
\\
Kan extensions
\\
Homotopical Algebra : Theory of Model Categories
\\
Topoi
\\
\section{Practical Category Theory : David Koontz}
Monoids, SemiGroups
\\
\section{SemiGroup}
class Semigroup a where
\\
append :: $a$ $\rightarrow a \rightarrow a$
\\
Append is Associative
\\
$(a <> b) <> c == a <> (b <> c)$
\\
(A appended with B) appended with C 
\\
is the same as 
\\
A appended with (B appended with C)
\\
Monoid
\\
class (Semigroup a) <= Monoid a where 
\\
empty :: a 
\\
SemiGroup is append + associativity law
\\
Monoid is semigroup and it has an empty value
\\
\\
Monoid Laws
\\
The empty value doesn't change the meaning of a monoidal value
\\
a <> empty == a
\\
empty <> a == a
\\
(left and right identity)
\\
\\
\section{Higher Kinded Types : David Koontz}
Constraints
\\
Generic in structure
\\
\section{Functor}
f a
\\
Apply a function to the value(s) in the box
\\
map : Functor $f$ => $(a \rightarrow b) \rightarrow f$ $ a \rightarrow f$ $b$
\\

\end{document}
